% !TEX TS-program = pdflatex
% !TEX encoding = UTF-8 Unicode
% arara: pdflatex: { synctex: true }
% \listfiles
\pdfminorversion=7
\documentclass[a4paper,welsh,british,twocolumn]{article}
% !TEX TS-program = pdflatex
% !TEX encoding = UTF-8 Unicode
\usepackage{svn-prov}
\ProvidesFileSVN{$Id: handouts-da.cfg 5795 2017-01-30 18:09:36Z cfrees $}[\filebase: configuration for supplementary handouts \revinfo]
\usepackage{svn-multi}
	\svnidlong
		{$HeadURL: file:///mnt/between/svn/dysgu/trunk/DoctoralAcademy/config/handouts-da.cfg $}
		{$LastChangedBy: cfrees $}
		{$LastChangedRevision: 5795 $}
		{$LastChangedDate: 2017-01-30 18:09:36 +0000 (Llu, 30 Ion 2017) $}
% best not to mix \svnid and \svnidlong (or to always typeset time-zone with time? see documentation)
%	\svnid{$Id: handouts-da.cfg 5795 2017-01-30 18:09:36Z cfrees $}
	\svnRegisterAuthor{cfrees}{Clea F. Rees}
\usepackage{babel}
\usepackage[utf8]{inputenc}
\usepackage[tt=lining]{cfr-lm}
\usepackage{enumitem,geometry,url,fancyhdr,fancyref}
\usepackage{csquotes}
\MakeAutoQuote{‘}{’}
\MakeAutoQuote*{“}{”}
\geometry{scale=.9}
\setlength{\columnseprule}{0.4pt}
\urlstyle{sf}

\providecommand{\cymraeg}[1]{\foreignlanguage{welsh}{#1}}
\providecommand{\welsh}[1]{\foreignlanguage{welsh}{#1}}

\fancyhf{}
\renewcommand*\headrulewidth{0pt}
\pagestyle{fancy}

\author{Clea F. Rees}
\date{}

\AtBeginDocument{%
  \pdfinfo{%
    /Author   (Clea F. Rees)
    /Subject  (LaTeX)}
  \pdfcatalog{%
    /URL      (http://cfrees.wordpress.com/)
    /PageMode /UseOutlines}   % other values: /UseNone, /UseOutlines, /UseThumbs, /FullScreen
  %[openaction <actionspec>]
}


\endinput

\svnidlong
  {$HeadURL: file:///mnt/between/svn/dysgu/trunk/DoctoralAcademy/handouts/latex-gweithdy-packages.tex $}
  {$LastChangedBy: cfrees $}
  {$LastChangedRevision: 5802 $}
  {$LastChangedDate: 2017-01-31 04:26:20 +0000 (Maw, 31 Ion 2017) $}
  % best not to mix \svnid and \svnidlong (or to always typeset time-zone with time? see documentation)
  %	\svnid{$Id: latex-gweithdy-packages.tex 5802 2017-01-31 04:26:20Z cfrees $}
\title{\LaTeX{} Package Recommendations}
\fancyhf[cf]{%
  Find packages in the Comprehensive \TeX{} Archive Network (CTAN) at \url{ctan.org}.
  Browse by topic at \url{ctan.org/topic}.}
\begin{document}
\pdfinfo{%
  /Title	(LaTeX Package Recommendations)
  /Keywords	(LaTeX, package)}
\maketitle\thispagestyle{fancy}
\newlist{pkgdescription}{description}{1}
\setlist[pkgdescription]{font=\bfseries\ttfamily}
\newcommand*\lpack[1]{\texttt{\bfseries #1}}
\section{General}
You should almost always use:
\begin{pkgdescription}
  \item[babel] Pass \verb|welsh,british| to your class.
  \item[inputenc] Load with option \verb|utf8|.
  \item[fontenc] Load with option \verb|T1|.
  \item[textcomp]
  \item[microtype]
\end{pkgdescription}
\section{Document Layout}
If you are using a standard class (e.g.\ \lpack{article}, \lpack{book} or \lpack{report}):
\begin{pkgdescription}
  \item[geometry] to change page dimensions.
  \item[fancyhdr] for custom headers/footers.
  \item[footmisc] for customised footnotes.
  \item[titling] to customise title and use document metadata after \verb|\maketitle|.
  \item[titlesec] for custom sectioning and \lpack{titleps} for headers/footers.
\end{pkgdescription}
\section{Mathematics}
\begin{pkgdescription}
  \item[mathtools] for enhanced \lpack{amsmath}.
  \item[amssymb] for more symbols, scripts.
  \item[ntheorem] for enhanced theorem environments.
\end{pkgdescription}
\section{Quotes \& Quoting}
\begin{pkgdescription}
  \item[csquotes] for context- and language-sensitive quotations and quotation marks. Recommended if using \lpack{biblatex}.
\end{pkgdescription}
\section{Citations \& Bibliographies}
\begin{pkgdescription}
  \item[biblatex] Load with option \verb|backend=biber|.
\end{pkgdescription}
\section{Cross-Referencing}
\begin{pkgdescription}
  \item[fancyref] for enhanced cross-references.
  \item[cleverref] for enhanced cross-references.
\end{pkgdescription}
\section{Lists}
\begin{pkgdescription}
  \item[enumitem] for custom lists.
  \item[glossaries] for glossaries and lists of acronyms.
\end{pkgdescription}
\section{Tables}
\begin{pkgdescription}
  \item[array] for enhanced tabular environments.
  \item[booktabs] for professional quality tables.
  \item[longtable] for multi-page tables.
  \item[tabularx] for tables with specified width.
  \item[threeparttablex] for tables with notes.
  \item[multirow] for cells spanning multiple rows.
\end{pkgdescription}
\section{Floats}
\begin{pkgdescription}
  \item[caption] to customise captions.
  \item[float] more options for floats.
  \item[placeins] to control float placement.
  \item[subcaption] for sub-figures, sub-tables and sub-captions.
  \item[floatrow] for aligned sub-figures.
  \item[rotating] to rotate floats.
\end{pkgdescription}
\section{Hyperlinks}
\begin{pkgdescription}
  \item[hyperref] for hyperlinks.
  \item[bookmark] for enhanced bookmarks.
\end{pkgdescription}
\section{Images \& Colour}
\begin{pkgdescription}
  \item[graphicx] to load external images.
  \item[xcolor] for colour.
\end{pkgdescription}
\section{Diagrams}
\begin{pkgdescription}
  \item[tikz] for diagrams.
  \emph{Many} specialised extensions available.
  \item[pgfplots] for plots.
  Includes \lpack{pgfplotstable} for data tables.
\end{pkgdescription}
\section{External Data}
\begin{pkgdescription}
  \item[datatool] for data manipulation.
  \item[textmerg] for merging text.
\end{pkgdescription}
\section{Version Control}
\begin{pkgdescription}
  \item[svn-multi] for use with \verb|subversion|.
  \item[gitinfo2] for use with \verb|git|.
\end{pkgdescription}
\appendix
\newcommand*\seesci{\emph{See also \fref{sec:sci}.}}
\section{Biology}
\seesci
\begin{pkgdescription}
  \item[bracketkey] for bracketed identification keys.
  \item[dichokey] for dichotomous identification keys.
  \item[shipunov] for identification keys, classification lists and more.
  \item[texshade] for nucleotide and peptide alignments.
  Can process alignments in \textsc{msf}, \textsc{aln} and \textsc{fasta} formats.
  \item[textopo] for shaded membrane protein topology plots.
\end{pkgdescription}
\section{Chemistry}
\seesci
% Ref.: \url{}
\begin{pkgdescription}
  \item[chemfig] for reaction schemes and \lpack{tikz}-based molecules.
  \lpack{chemformula} and \lpack{mhchem} are alternatives.
  \item[chemformula] for formulae and reactions.
  \item[mhchem] for formulae and equations.
  Includes \lpack{hpstatement} (H and P Statements) and \lpack{rsphrase} (R and S Phrases).
  \item[modiagram] for \lpack{tikz}-based molecular orbital diagrams.
  \item[tikzorbital] for \lpack{tikz}-based molecular orbitals, inc.\ s, p and d.
\end{pkgdescription}
\section{Computer Science}
\emph{See also sections \ref{sec:logic} and \ref{sec:sci}.}
% Ref.: \url{}
\begin{pkgdescription}
  \item[algorithms] for pseudo-code.
  \item[algorithm2e] for floating pseudo-code.
  \item[listings] for source code.
  \item[minted] for highlighted source code.
  \item[pgf-umlcd] and \lpack{pgf-umlsd} for \lpack{tikz}-based UML diagrams.
  \item[sa-tikz] for \lpack{tikz}-based switching architectures.
\end{pkgdescription}
\section{Engineering}\label{sec:eng}
\emph{See also sections \ref{sec:phys} and \ref{sec:sci}.}
% Ref.: \url{}
\begin{pkgdescription}
  \item[bloques] for simple \lpack{tikz}-based control diagrams.
  \item[circuitikz] for \lpack{tikz}-based electrical and electronic circuits.
\end{pkgdescription}
\section{Humanities}
% Ref.: \url{}
\begin{pkgdescription}
  \item[bibleref] for referencing and indexing Bible verses.
  \item[classics] to cite classic works sensibly.
  \item[reledmac] for critical editions and \lpack{reledpar} for parallel texts.
  \item[ednotes] for critical editions of handwritten manuscripts.
  \item[edfnotes] for critical editions of printed texts with footnotes.
  \item[handout] for handouts consisting of textual excerpts.
  \item[poemscol] for critical editions of poetry.
  \item[schemata] for topical schemata of the kind sometimes used to illustrate conceptual structure e.g.~in Scholastic thought.
  \item[verse] for verse without annotations.
\end{pkgdescription}
\section{Linguistics}
% Ref.: \url{www.essex.ac.uk/linguistics/external/clmt/latex4ling}
\begin{pkgdescription}
  \item[gb4e] for examples, glosses etc.
  \item[expex] for enhanced examples, glosses etc.
  \item[leipzig] for standard and custom glossing abbreviations.
  \item[ot-tableau] for optimality-theoretic tableaux.
  \item[qtree] for syntactic trees without \lpack{tikz}.
  \item[forest] for enhanced \lpack{tikz}-based syntactic trees.
  \item[stmaryrd] for semantics brackets.
  \item[tikz-dependency] for \lpack{tikz}-based dependency graphs.
  \item[tipa] for IPA fonts. But consider Xe\LaTeX{} or Lua\LaTeX.
\end{pkgdescription}
\section{Logic}\label{sec:logic}
Ref.: \url{www.latexforlogicians.net}
\begin{pkgdescription}
  \item[algorithms] for algorithms.
  \item[gene-logic] for better spacing of maths symbols.
  \item[bussproofs] for natural deduction/Gentzen sequent proofs.
  \item[prftree] a newer alternative to \lpack{bussproofs}.
  \item[lplfitch] for ‘Fitch’-style proofs.
  \item[logicproof] for ‘Fitch’-style proofs with boxed sub-proofs.
  \item[natded] for Jaśkowski-/Kalish-Montague-style proofs.
  \item[qtree] for proof trees without \lpack{tikz}.
  \item[prooftrees] for enhanced \lpack{forest}/\lpack{tikz}-based proof trees.
  \item[tikz-cd] for \lpack{tikz}-based commutative diagrams.
  \item[tikz-inet] for \lpack{tikz}-based interaction nets.
  \item[turnstile] for turnstiles of all kinds.
\end{pkgdescription}
\section{Physics}\label{sec:phys}
\emph{See also sections \ref{sec:eng} and \ref{sec:sci}.}
% Ref.: \url{}
\begin{pkgdescription}
  \item[feyn] for inline Feynman diagrams.
  \item[feynmp] or \lpack{feynmf} for Feynman diagrams.
\end{pkgdescription}
\section{Sciences}\label{sec:sci}
% Ref.: \url{}
\begin{pkgdescription}
  \item[miller] for Miller indices.
  \item[siunitx] for SI units.
\end{pkgdescription}
\section*{Additional Symbols}\label{sec:symbols}
\begin{pkgdescription}
  \item[comprehensive] provides a more-or-less comprehensive list.
  \item[adforn,adfarrows,adfbullets] for ADF symbol fonts.
  \item[dictsym] for dictionary symbols.
  \item[marvosym] for assorted symbols.
  \item[pifont] for Zapf Dingbats.
  \item[wasysym] for the symbol font \texttt{wasy}.
\end{pkgdescription}
\end{document}
